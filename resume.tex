\documentclass{resume} % Use the custom resume.cls style
\usepackage[left=0.75in,top=0.6in,right=0.75in,bottom=0.6in]{geometry} % Document margins
\usepackage{xspace}
\usepackage{xcolor}
\usepackage{hyperref}
\usepackage{gensymb}
\hypersetup{
  pdftitle={Nicolas Viennot -- Resume},
  pdfauthor={Nicolas Viennot},
  colorlinks=true,
  % citecolor=blue,
}

\renewcommand\UrlFont{\color{teal}\rmfamily\itshape}


\newcommand{\columbia}{{Columbia University}\xspace}
\newcommand{\supelec}{{Sup\'{e}lec}\xspace}
\newcommand{\eiffel}{{Lyc\'{e}e Gustave Eiffel}\xspace}

\name{Nicolas Viennot}
\address{nicolas@viennot.com \\ New York, NY \\ 646-504-6464}

\begin{document}

\vspace{-1em}

%%%%%%%%%%%%%%%%%
% Education
%%%%%%%%%%%%%%%%%

\begin{rSection}{Education}

\begin{rSubsection}{\columbia}{2010 -- Expected May. 2016}{Ph.D. in Computer Science}{New York, NY}
\item Advisor: Jason Nieh.
\item Thesis: Record-Replay Mechanisms Applied to Operating Systems and Distributed Systems.
\end{rSubsection}

\begin{rSubsection}{\columbia}{2007 -- 2009}{M.S. in Computer Science}{New York, NY}
\item Focus areas: Operating Systems, Security, Networking, Programming Languages.
\item GPA: 4.07/4.00
\end{rSubsection}

\begin{rSubsection}{\supelec}{2005 -- 2007}{Engineering Degree}{Paris, France}
\item \supelec is a top-ranked {\em Grande \'{E}cole} in France.
\item Focus areas: Computer Science, Signal Processing, Electronic, Power Engineering, and Economics.
\end{rSubsection}

% \begin{rSubsection}{\eiffel}{2003 -- 2005}{Preparatory School}{Bordeaux, France}
% \item Intensive two-year program preparing to Engineering Schools competitive entrance.
% \item Focus areas: Mathematics, Physics, System Engineering.
% \end{rSubsection}

\end{rSection}

%%%%%%%%%%%%%%%%%
% Experience
%%%%%%%%%%%%%%%%%

\begin{rSection}{Work Experience}

% \begin{rSubsection}{NimbleDroid}{2014 -- 2015}{Full Stack Developer}{New York, NY}
% \item XXXX
% \end{rSubsection}

\begin{rSubsection}{Crowdtap}{2012 -- 2013}{System Engineer}{New York, NY}
\item Implemented Synapse at Crowdtap, a startup with over 500,000 users. Synapse interconnects dozen of their services.
% \item Synapse is an easy-to-use, strong-semantics system for large-scale, data-driven Web service integration.
\item Synapse lets independent services cleanly share data with each other with 
an easy-to-use publish/subscribe API.
\item Synapse transparently synchronizes heterogeneous databases with causal consistency semantics in a scalable manner.
\end{rSubsection}

% \begin{rSubsection}{Cellrox}{2012}{System Engineer}{New York, NY}
% \item Cellrox provides mobile phone virtualization solutions. Developed a web-based tool that manages the fleet of devices.
% \item The tool was deployed on companies' premises, removing any hopes of using 3rd party services.
% \end{rSubsection}

% \begin{rSubsection}{SaiMaa}{Fall 2010}{Project Manager}{New York, NY}
% \item Implemented their event registration and bookstore systems to capture \$0.5M in revenue.
% \item Hired contractors to join forces on the project. Supervised developers with an Agile methodology.
% \end{rSubsection}

\begin{rSubsection}{ASDLabs}{2009 -- 2010}{Full Stack Developer}{New York, NY}
\item Involved in dozens of projects, primarily in the health-care and financial
      industries. Tasks included business analysis, project planning, application
      design and development. Often brought creative solutions to the table to cut costs.
\item Dramatically improved internal company processes by training employees to
      leverage project management tools including centralized file sharing and source revision control systems.
% \item Unsuccessful in converting the company's focus from PHP and .NET to Django or Rails.
\end{rSubsection}

\begin{rSubsection}{Columbia University}{Fall 2009}{Teaching Assistant}{New York, NY}
\item Lead Teaching Assistant in Prof. Jason Nieh's notoriously rigorous Operating System class (70 students).
% \item Brought innovative additions to the teaching workflow and student professor interactions by implementing revision
% control systems and virtual appliances.
\item Responsible for creating homework assignments, creating scoring rubrics, grading,
individual demos, managing repositories for students, and answering questions on an online board (posted more
than 1000 messages during the semester).
% \item Anonymous comments from students: \url{http://viennot.biz/os1.comments.txt}
\end{rSubsection}

\begin{rSubsection}{EDIMS}{Summer 2006}{Developer}{New York, NY}
\item Fixed memory leaks in the main product (2.5M lines of code)
      resulting in no longer having to reboot computers daily in hospitals.
      These leaks were discovered in 3 days. Previously hired contractors were unable to find them in 3 months.

      % previously hired contractors failed to isolate problem after 3 month endeavor.

\item Developed a patient chart editor for emergency rooms in hospitals.
\end{rSubsection}

\begin{rSubsection}{\supelec}{Spring 2006}{Software Engineer}{Paris, France}
\item Installed inexpensive servers to broadcast national TV channels coming
      from DVB-T adapters over the campus network using IP multicast,
      reaching over 600 students. This service is still running today.
\end{rSubsection}

\begin{rSubsection}{VIBI Prod.}{Winter 2004}{Hardware and Software Engineer}{Bordeaux, France}
\item Designed a radio interface for a USB tablet hidden in a clipboard to be used by professional magicians.
\item The main difficulty was to reverse engineer the tablet drivers to understand its proprietary tablet protocol.
\end{rSubsection}

\begin{rSubsection}{Elegenics Inc.}{Summer 2003}{Hardware and Software Engineer}{Mountain View, CA}
\item Designed the software stack in C++ and C\#. It controls a 5-axis
      robot, a bar-code reader, and a servomotor to select microscope lenses.
      It detects hundreds of petri dishes and places them under the
      microscope to capture micro-organisms.
\item Designed a custom electronic board to connect the different hardware components.
\end{rSubsection}

\end{rSection}

%%%%%%%%%%%%%%%%%
% Research Projects
%%%%%%%%%%%%%%%%%

\begin{rSection}{Personal and Research Projects}

\begin{rSubsection}{tmate.io}{2013 -- Present}{Instant Terminal Sharing}{\url{http://tmate.io/}}
\item tmate is a tmux fork (terminal multiplexer) coded in C. It allows to share
  terminals very easily and securely. tmate is mostly used for pair-programming.
      More than a thousand paired sessions are seen each week.
\item tmate is sponsored by DigitalOcean and deployed on 6 different geographic locations for optimal latency.
\end{rSubsection}

\clearpage

\begin{rSubsection}{NoBrainer}{2012 -- Present}{Ruby ORM for RethinkDB}{\url{http://nobrainer.io/}}
\item After too many frustrating experiences with ActiveRecord or Mongoid due to
  inconsistent APIs and weak semantics, I decided to make my own ORM (Object Relational Mapper).
  The design goal is to allow quick application prototyping, while providing a rock-solid production experience.
  API design is one of the most difficult aspect of the project.
\end{rSubsection}

\begin{rSubsection}{PlayDrone}{2012 -- 2014}{Google Play Crawler}{\url{https://github.com/nviennot/playdrone}}
\item PlayDrone is the first scalable Google Play crawler. We used it to analyze the source code of over 1.1M applications.
\item We discovered numerous OAuth credential leaks. We collaborated with Facebook,
    Google and Amazon to mitigate the issues, making the Google Play store a safer place.
\item PlayDrone was used at archive.org to provide a full dump of the Google Play store for other researchers.
\end{rSubsection}

\begin{rSubsection}{Scribe}{2008 -- 2012}{Deterministic Record-Replay Engine}{\url{https://github.com/nviennot/linux-2.6-scribe}}
\item Scribe is a transparent, deterministic application execution record-replay mechanism operating at the kernel level.
\item We used Scribe to build RacePro and Dora. Racepro is a system for detecting harmful process races in deployed systems.
Dora is a mutable replay engine which enables reproducing non-deterministic bugs with retroactive debugging.
\end{rSubsection}

\begin{rSubsection}{Reflow Oven}{2004 -- 2005}{\url{http://viennot.com/reflow\_oven.pdf}}{}
\item Designed a reflow oven used to solder components on a circuit board from scratch.
\item The oven features 2kW of infrared heat, allowing surfaces to reach
      $200\degree$C in 30s. The heating elements are controlled by a standalone
      electronic board that enforces a precise temperature profile with a PID feedback loop.
\end{rSubsection}

% \item reverse engineering
% \item github, contributed to numerous open source proj
\end{rSection}

%%%%%%%%%%%%%%%%%
% Awards and Patents
%%%%%%%%%%%%%%%%%

\begin{rSection}{Awards and Patents}
\begin{rList}
\item Kenneth C. Sevcik Outstanding Student Paper Award for ``{\em A Measurement Study of Google Play}'' (SIGMETRICS '14).
\item US Patent 8402318: Systems and methods for recording and replaying application execution.
\end{rList}
\end{rSection}

%%%%%%%%%%%%%%%%%
% Publications
%%%%%%%%%%%%%%%%%

\begin{rSection}{Selected Publications}
\newcounter{itemcounter}
\begin{list}{\arabic{itemcounter}.}{\usecounter{itemcounter}\leftmargin=0em}
\newenvironment{pub}[5]{ {\bf #1} \\ #2, {\em #3 (#4)}, #5}

\item \pub{Synapse: A Microservices Architecture for Heterogeneous-Database Web Applications}
{Nicolas Viennot, Mathias L\'{e}cuyer, Jonathan Bell, Roxana Geambasu, Jason Nieh}
{Proceedings of the 10th European Conference on Computer Systems}
{EuroSys '15}{Bordeaux, France, April 2015.}\\
Synapse was also presented at RubyConf'13, Miami, FL, Nov. 2013.

\item \pub{A Measurement Study of Google Play}
{Nicolas Viennot, Edward Garcia, and Jason Nieh}
{Proceedings of the 2014 ACM International Conference on Measurement and Modeling of Computer Systems}
{SIGMETRICS '14}{Austin, TX, June 2014. (Kenneth C. Sevcik Outstanding Student Paper Award.)}

% \item \pub{Cider: Native Execution of iOS Apps on Android}
% {Jeremy Andrus, Alexander Van't Hof, Naser AlDuaij, Christoffer Dall, Nicolas Viennot, and Jason Nieh}
% {Proceedings of the 19th International Conference on Architectural Support for Programming Languages and Operating Systems}
% {ASPLOS '14}{Salt Lake City, UT, March 2014.}

\item \pub{Transparent Mutable Replay for Multicore Debugging and Patch Validation}
{Nicolas Viennot, Sid Nair, and Jason Nieh}
{Proceedings of the 18th International Conference on Architectural Support for Programming Languages and Operating Systems}
{ASPLOS '13}{Houston, TX, March 2013.}

\item \pub{Pervasive Detection of Process Races in Deployed Systems}
{Oren Laadan, Nicolas Viennot, Chia-Che Tsai, Chris Blinn, Junfeng Yang, and Jason Nieh}
{Proceedings of the 23rd ACM Symposium on Operating Systems Principles}
{SOSP '11}{Cascais, Portugal, October 2011.}

\item \pub{Transparent, Lightweight Application Execution Replay on Commodity Multiprocessor Operating Systems}
{Oren Laadan, Nicolas Viennot, Jason Nieh}
{Proceedings of the 2010 ACM International Conference on Measurement and Modeling of Computer Systems}
{SIGMETRICS '10}{New York, NY, June 2010.}

\item \pub{ASSURE: Automatic Software Self-healing Using REscue points}
{Stelios Sidiroglou, Oren Laadan, Carlos Perez, Nicolas Viennot, Jason Nieh, Angelos D. Keromytis}
{Proceedings of the 14th International Conference on Architectural Support for Programming Languages and Operating Systems}
{ASPLOS '09}{Washington, DC, March 2009.}

\end{list}
\end{rSection}

%%%%%%%%%%%%%%%%%
% Skills
%%%%%%%%%%%%%%%%%

\begin{rSection}{Skills}
\begin{rList}
\item Strongest programming languages: Ruby, C, JavaScript, HTML/CSS, x86 assembly. (GitHub: \url{https://github.com/nviennot})
\item Hobbies: Photography with a focus on headshot and street photography.
\item Native French speaker, fluent in English.
\end{rList}
\end{rSection}

\end{document}
